\documentclass[man]{apa6}
\usepackage{lmodern}
\usepackage{amssymb,amsmath}
\usepackage{ifxetex,ifluatex}
\usepackage{fixltx2e} % provides \textsubscript
\ifnum 0\ifxetex 1\fi\ifluatex 1\fi=0 % if pdftex
  \usepackage[T1]{fontenc}
  \usepackage[utf8]{inputenc}
\else % if luatex or xelatex
  \ifxetex
    \usepackage{mathspec}
  \else
    \usepackage{fontspec}
  \fi
  \defaultfontfeatures{Ligatures=TeX,Scale=MatchLowercase}
\fi
% use upquote if available, for straight quotes in verbatim environments
\IfFileExists{upquote.sty}{\usepackage{upquote}}{}
% use microtype if available
\IfFileExists{microtype.sty}{%
\usepackage{microtype}
\UseMicrotypeSet[protrusion]{basicmath} % disable protrusion for tt fonts
}{}
\usepackage{hyperref}
\hypersetup{unicode=true,
            pdftitle={A Logistic Regression Approach to CoIL Challenge 2000},
            pdfauthor={Corey Arnouts, Adam Douglas, Jason Givens-Doyle, \& Michael Silva},
            pdfkeywords={CoIL Challenge, Logistic Regression},
            pdfborder={0 0 0},
            breaklinks=true}
\urlstyle{same}  % don't use monospace font for urls
\usepackage{graphicx,grffile}
\makeatletter
\def\maxwidth{\ifdim\Gin@nat@width>\linewidth\linewidth\else\Gin@nat@width\fi}
\def\maxheight{\ifdim\Gin@nat@height>\textheight\textheight\else\Gin@nat@height\fi}
\makeatother
% Scale images if necessary, so that they will not overflow the page
% margins by default, and it is still possible to overwrite the defaults
% using explicit options in \includegraphics[width, height, ...]{}
\setkeys{Gin}{width=\maxwidth,height=\maxheight,keepaspectratio}
\IfFileExists{parskip.sty}{%
\usepackage{parskip}
}{% else
\setlength{\parindent}{0pt}
\setlength{\parskip}{6pt plus 2pt minus 1pt}
}
\setlength{\emergencystretch}{3em}  % prevent overfull lines
\providecommand{\tightlist}{%
  \setlength{\itemsep}{0pt}\setlength{\parskip}{0pt}}
\setcounter{secnumdepth}{0}
% Redefines (sub)paragraphs to behave more like sections
\ifx\paragraph\undefined\else
\let\oldparagraph\paragraph
\renewcommand{\paragraph}[1]{\oldparagraph{#1}\mbox{}}
\fi
\ifx\subparagraph\undefined\else
\let\oldsubparagraph\subparagraph
\renewcommand{\subparagraph}[1]{\oldsubparagraph{#1}\mbox{}}
\fi

%%% Use protect on footnotes to avoid problems with footnotes in titles
\let\rmarkdownfootnote\footnote%
\def\footnote{\protect\rmarkdownfootnote}


  \title{A Logistic Regression Approach to CoIL Challenge 2000}
    \author{Corey Arnouts\textsuperscript{1}, Adam Douglas\textsuperscript{1}, Jason Givens-Doyle\textsuperscript{1}, \& Michael Silva\textsuperscript{1}}
    \date{}
  
\shorttitle{ }
\affiliation{
\vspace{0.5cm}
\textsuperscript{1} MS in Data Science Students CUNY School of Professional Studies}
\keywords{CoIL Challenge, Logistic Regression\newline\indent Word count: X}
\usepackage{csquotes}
\usepackage{upgreek}
\captionsetup{font=singlespacing,justification=justified}

\usepackage{longtable}
\usepackage{lscape}
\usepackage{multirow}
\usepackage{tabularx}
\usepackage[flushleft]{threeparttable}
\usepackage{threeparttablex}

\newenvironment{lltable}{\begin{landscape}\begin{center}\begin{ThreePartTable}}{\end{ThreePartTable}\end{center}\end{landscape}}

\makeatletter
\newcommand\LastLTentrywidth{1em}
\newlength\longtablewidth
\setlength{\longtablewidth}{1in}
\newcommand{\getlongtablewidth}{\begingroup \ifcsname LT@\roman{LT@tables}\endcsname \global\longtablewidth=0pt \renewcommand{\LT@entry}[2]{\global\advance\longtablewidth by ##2\relax\gdef\LastLTentrywidth{##2}}\@nameuse{LT@\roman{LT@tables}} \fi \endgroup}


\DeclareDelayedFloatFlavor{ThreePartTable}{table}
\DeclareDelayedFloatFlavor{lltable}{table}
\DeclareDelayedFloatFlavor*{longtable}{table}
\makeatletter
\renewcommand{\efloat@iwrite}[1]{\immediate\expandafter\protected@write\csname efloat@post#1\endcsname{}}
\makeatother

\authornote{

Correspondence concerning this article should be addressed to Corey Arnouts, 119 W 31st St., New York, NY 10001. E-mail: \href{mailto:Corey.Arnouts@spsmail.cuny.edu}{\nolinkurl{Corey.Arnouts@spsmail.cuny.edu}}}

\abstract{
A logistic regression based solution to the CoIL Challenge 2000 is described. The challenge consists of correctly identifying potential customers for an insurance product, and describing their characteristics.


}

\begin{document}
\maketitle

\hypertarget{introduction}{%
\section{Introduction}\label{introduction}}

Businesses use data science to extract insights from data. It has many practical business applications. Identifying households to include in a marketing campaign is one application. One example using real world data is the Computational Intelligence and Learning (CoIL) Challenge. The CoIL Challenge competition was held from March 17 to May 8 in 2000. The challenge is to:

\begin{enumerate}
\def\labelenumi{\arabic{enumi}.}
\item
  Identify potential customers for an insurance policy; and
\item
  Provide a description of this customer base.
\end{enumerate}

In total 147 participants registered and 43 submitted solutions (Putten, Ruiter, \& Someren, 2000). In this paper we set out to complete the first part of the COIL Challenge. \textbf{SUMARISE FINDINGS?}

\hypertarget{literature-review}{%
\section{Literature Review}\label{literature-review}}

Participants used a variety of approaches in formulating their submissions including: Boosted Decision Tree (McKone \& Stenger, 2000), Classification and Regressio Tree (CART) (Simmonds, 2000), Classification Trees with Bagging (White \& Liu, 2000), C4.5 (Rickets, 2000; Seewald, 2000), Evolutionary Algorithm (Koudijs, 2000), Fuzzy Classifier (János Abonyi, 2000; Kaymak \& Setnes, 2000), Genetic Algorithms and Hill-climbers (Carter, 2000), Inductive Learning by Logic Minimization (ILLM) (Gamberger, 2000; Šmuc, 2000), Instance Based Reasoning (iBARET) (Mikšovský \& Klema, 2000), K-Means (Vesanto \& Sinkkonen, 2000), KXEN (Bera \& Lamy, 2000), LOGIT (Doornik \& Moyle, 2000), Mask Perceptron with Boosting (Leckie \& Ferra, 2000), Midos Algorithm (Krogel, 2000), N-Tuple Classifier (Jorgensen \& Linneberg, 2000), Naïve Bayes (Elkan, 2000; Kontkanen, 2000), Neural Networks(Brierley, 2000; Crocoll, 2000; Kim \& Street, 2000; Shtovba \& Mashnitskiy, 2000), Phase Intervals and Genetic Algorithms (Shtovba, 2000), Scoring System (Lewandowski, 2000), Support Vector Machines(Keerthi \& Ong, 2000), and XCS (Greenyer, 2000).

\includegraphics{Manuscript_files/figure-latex/unnamed-chunk-1-1.pdf}

The maximum number of policyowners that could be found was 238. The submissions identified 95 policy owners on average. The winning model (Elkan, 2000) identified 121 policy owners. Random selection results in identifying 42 policy owners. The standard benchmark tests result in 94 (k-nearest neighbor), 102 (naïve bayes), 105 (neural networks) and 118 (linear) policy owners. (Putten et al., 2000).

\hypertarget{methodology}{%
\section{Methodology}\label{methodology}}

\hypertarget{experimentation-and-results}{%
\section{Experimentation and Results}\label{experimentation-and-results}}

\hypertarget{discussion-and-conclusions}{%
\section{Discussion and Conclusions}\label{discussion-and-conclusions}}

\newpage

\hypertarget{references}{%
\section{References}\label{references}}

\begingroup
\setlength{\parindent}{-0.5in}
\setlength{\leftskip}{0.5in}

\hypertarget{refs}{}
\leavevmode\hypertarget{ref-Bera}{}%
Bera, M., \& Lamy, B. (2000). Kxen at coil challenge 2000. Retrieved from \url{http://liacs.leidenuniv.nl/~puttenpwhvander/library/cc2000/BERAPS~1.pdf}

\leavevmode\hypertarget{ref-Brierley}{}%
Brierley, P. (2000). COIL 2000 challenge: Characteristics of caravan insurance policy owners. Retrieved from \url{http://liacs.leidenuniv.nl/~puttenpwhvander/library/cc2000/BRIERL~1.pdf}

\leavevmode\hypertarget{ref-Carter}{}%
Carter, J. (2000). Coil 2000 challenge submission: Genetic algorithms and hill-climbers. Retrieved from \url{http://liacs.leidenuniv.nl/~puttenpwhvander/library/cc2000/CARTER~1.pdf}

\leavevmode\hypertarget{ref-Crocoll}{}%
Crocoll, W. M. (2000). Artificial neural network portion of coil study. Retrieved from \url{http://www.liacs.nl/~putten/library/cc2000/CROCOL~1.pdf}

\leavevmode\hypertarget{ref-Doornik}{}%
Doornik, J. A., \& Moyle, S. (2000). LOGIT modelling. Retrieved from \url{http://liacs.leidenuniv.nl/~puttenpwhvander/library/cc2000/MOYLEP~1.pdf}

\leavevmode\hypertarget{ref-Elkan}{}%
Elkan, C. (2000). CoIL challenge 2000 entry. Retrieved from \url{http://liacs.leidenuniv.nl/~puttenpwhvander/library/cc2000/ELKANP~1.pdf}

\leavevmode\hypertarget{ref-Gamberger}{}%
Gamberger, D. (2000). Solution based on illm confirmation rule. Retrieved from \url{http://liacs.leidenuniv.nl/~puttenpwhvander/library/cc2000/GAMBER~1.pdf}

\leavevmode\hypertarget{ref-Greenyer}{}%
Greenyer, A. (2000). Coil 2000 competition. The use of a learning classifier system jxcs. Retrieved from \url{http://www.liacs.nl/~putten/library/cc2000/GREENY~1.pdf}

\leavevmode\hypertarget{ref-Abonyi}{}%
János Abonyi, H. R. (2000). A simple fuzzy classifier based on inconsistency analysis of labeled data. Retrieved from \url{http://liacs.leidenuniv.nl/~puttenpwhvander/library/cc2000/ABONYI~1.pdf}

\leavevmode\hypertarget{ref-Jorgensen}{}%
Jorgensen, T. M., \& Linneberg, C. (2000). Subspace projections -- an approach to variable selection and modeling. Retrieved from \url{http://liacs.leidenuniv.nl/~puttenpwhvander/library/cc2000/JORGEN~1.pdf}

\leavevmode\hypertarget{ref-Kaymak}{}%
Kaymak, U., \& Setnes, M. (2000). Target selection based on fuzzy clustering: A volume prototype approach to coil challenge 2000. Retrieved from \url{http://liacs.leidenuniv.nl/~puttenpwhvander/library/cc2000/KAYMAK~1.pdf}

\leavevmode\hypertarget{ref-Keerthi}{}%
Keerthi, S. S., \& Ong, C. J. (2000). Solution of the coil challenge 2000 task using support vector machines. Retrieved from \url{http://liacs.leidenuniv.nl/~puttenpwhvander/library/cc2000/KEERTH~1.pdf}

\leavevmode\hypertarget{ref-Kim}{}%
Kim, Y., \& Street, W. N. (2000). CoIL challenge 2000: Choosing and explaining likely caravan insurance customers. Retrieved from \url{http://liacs.leidenuniv.nl/~puttenpwhvander/library/cc2000/STREET~1.pdf}

\leavevmode\hypertarget{ref-Kontkanen}{}%
Kontkanen, P. (2000). CoIL 2000 submission. Retrieved from \url{http://liacs.leidenuniv.nl/~puttenpwhvander/library/cc2000/KONTKA~1.pdf}

\leavevmode\hypertarget{ref-Koudijs}{}%
Koudijs, A. (2000). CoIL challenge 2000 submission for the description task. Retrieved from \url{http://www.liacs.nl/~putten/library/cc2000/KOUDIJ~1.pdf}

\leavevmode\hypertarget{ref-Krogel}{}%
Krogel, M.-A. (2000). A data mining case study. Retrieved from \url{http://www.liacs.nl/~putten/library/cc2000/KROGEL~1.pdf}

\leavevmode\hypertarget{ref-Leckie}{}%
Leckie, C., \& Ferra, H. (2000). COIL challenge 2000 description task. Retrieved from \url{http://www.liacs.nl/~putten/library/cc2000/LECKIE~1.pdf}

\leavevmode\hypertarget{ref-Lewandowski}{}%
Lewandowski, A. (2000). How to detect potential customers. Retrieved from \url{http://www.liacs.nl/~putten/library/cc2000/LEWAND~1.pdf}

\leavevmode\hypertarget{ref-McKone}{}%
McKone, T., \& Stenger, C. (2000). COIL challenge 2000 submission. Retrieved from \url{http://liacs.leidenuniv.nl/~puttenpwhvander/library/cc2000/MCKONE~1.pdf}

\leavevmode\hypertarget{ref-Miksovsky}{}%
Mikšovský, P., \& Klema, J. (2000). CoIL challenge 2000. Retrieved from \url{http://liacs.leidenuniv.nl/~puttenpwhvander/library/cc2000/MIKSOV~1.pdf}

\leavevmode\hypertarget{ref-Putten}{}%
Putten, P., Ruiter, M., \& Someren, M. (2000). CoIL challenge 2000 tasks and results: Predicting and explaining caravan policy ownership. Retrieved from \url{http://liacs.leidenuniv.nl/~puttenpwhvander/library/cc2000/PUTTEN~1.pdf}

\leavevmode\hypertarget{ref-Rickets}{}%
Rickets, P. (2000). CoIL challenge 2000 submission. Retrieved from \url{http://liacs.leidenuniv.nl/~puttenpwhvander/library/cc2000/RICKET~1.pdf}

\leavevmode\hypertarget{ref-Seewald}{}%
Seewald, A. (2000). CoIL challenge 2000 submitted solution. Retrieved from \url{http://www.liacs.nl/~putten/library/cc2000/SEEWAL~1.pdf}

\leavevmode\hypertarget{ref-Shtovba}{}%
Shtovba, S. (2000). Phase intervals and genetic algorithms based competition task solution. Retrieved from \url{http://liacs.leidenuniv.nl/~puttenpwhvander/library/cc2000/SHTOVB~2.pdf}

\leavevmode\hypertarget{ref-Shtovba_Mashnitskiy}{}%
Shtovba, S., \& Mashnitskiy, Y. (2000). The backpropagation multilayer feedforward neural network based competition task solution. Retrieved from \url{http://liacs.leidenuniv.nl/~puttenpwhvander/library/cc2000/SHTOVB~1.pdf}

\leavevmode\hypertarget{ref-Simmonds}{}%
Simmonds, R. M. (2000). ACT study report using classification and regression tree (cart) analysis. Retrieved from \url{http://www.liacs.nl/~putten/library/cc2000/SIMMON~1.pdf}

\leavevmode\hypertarget{ref-Smuc}{}%
Šmuc, T. (2000). COIL 2000 challenge solution based on illm-sg methodology. Retrieved from \url{http://liacs.leidenuniv.nl/~puttenpwhvander/library/cc2000/SMUCPS~1.pdf}

\leavevmode\hypertarget{ref-Vesanto}{}%
Vesanto, J., \& Sinkkonen, J. (2000). Submission for the coil challenge 2000. Retrieved from \url{http://liacs.leidenuniv.nl/~puttenpwhvander/library/cc2000/VESANT~1.pdf}

\leavevmode\hypertarget{ref-White}{}%
White, A. P., \& Liu, W. Z. (2000). The coil challenge: An application of classification trees with bootstrap aggregation. Retrieved from \url{http://www.liacs.nl/~putten/library/cc2000/WHITEP~1.pdf}

\endgroup

\newpage

\hypertarget{appendices}{%
\section{Appendices}\label{appendices}}

\hypertarget{r-statistical-programming-code.}{%
\subsection{R statistical programming code.}\label{r-statistical-programming-code.}}

\begin{verbatim}
# CoIL Challenge Source Code 
library(tidyverse) 
# Download the data sets from UCI if they are not present 
url <- "https://archive.ics.uci.edu/ml/machine-learning-databases/tic-mld/" 
files <- c("ticdata2000.txt", "ticeval2000.txt", "tictgts2000.txt") 
for (file_name in files) { 
  file_path <- paste0("data/", file_name) 
  file_url <- paste0(url, file_name) 
  if (!file.exists(file_path)) { 
    message(paste("Downloading", file_name)) 
    download.file(file_url, file_path) 
  } 
} 
# Read in the data 
df <- read.delim("data/ticdata2000.txt", header = FALSE) 
names(df) <- c("MOSTYPE", "MAANTHUI", "MGEMOMV", "MGEMLEEF", "MOSHOOFD", 
               "MGODRK", "MGODPR", "MGODOV", "MGODGE", "MRELGE", "MRELSA", 
               "MRELOV", "MFALLEEN", "MFGEKIND", "MFWEKIND", "MOPLHOOG", 
               "MOPLMIDD", "MOPLLAAG", "MBERHOOG", "MBERZELF", "MBERBOER", 
               "MBERMIDD", "MBERARBG", "MBERARBO", "MSKA", "MSKB1", "MSKB2", 
               "MSKC", "MSKD", "MHHUUR", "MHKOOP", "MAUT1", "MAUT2", "MAUT0", 
               "MZFONDS", "MZPART", "MINKM30", "MINK3045", "MINK4575", 
               "MINK7512", "MINK123M", "MINKGEM", "MKOOPKLA", "PWAPART", 
               "PWABEDR", "PWALAND", "PPERSAUT", "PBESAUT", "PMOTSCO", 
               "PVRAAUT",  "PAANHANG", "PTRACTOR", "PWERKT", "PBROM", "PLEVEN", 
               "PPERSONG", "PGEZONG", "PWAOREG", "PBRAND", "PZEILPL", 
               "PPLEZIER", "PFIETS", "PINBOED", "PBYSTAND", "AWAPART", 
               "AWABEDR", "AWALAND", "APERSAUT", "ABESAUT", "AMOTSCO", 
               "AVRAAUT", "AAANHANG", "ATRACTOR", "AWERKT", "ABROM", "ALEVEN", 
               "APERSONG", "AGEZONG", "AWAOREG",  "ABRAND", "AZEILPL", 
               "APLEZIER", "AFIETS", "AINBOED", "ABYSTAND", "CARAVAN") 
eval <- read.delim("data/ticeval2000.txt", header = FALSE) 
temp <- read.delim("data/tictgts2000.txt", header = FALSE) 
eval$CARAVAN <- temp$V1 
names(eval) <- names(df) 
\end{verbatim}


\end{document}
